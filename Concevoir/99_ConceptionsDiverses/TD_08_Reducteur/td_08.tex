\documentclass[11pt,oneside]{article}
\input{coursHeadings}
\usepackage[final]{pdfpages} 
\usepackage[%
    pdftitle={TD Conception},
    pdfauthor={Xavier Pessoles},
    colorlinks=true,
    linkcolor=blue,
    citecolor=magenta]{hyperref}



% \makeatletter \let\ps@plain\ps@empty \makeatother
%% DEBUT DU DOCUMENT
%% =================
\sloppy
\hyphenpenalty 10000

\newcommand{\Pointilles}[1][3]{%
\multido{}{#1}{\makebox[\linewidth]{\dotfill}\\[\parskip]
}}


\begin{document}


\newboolean{prof}
\setboolean{prof}{false}
%------------- En tetes et Pieds de Pages ------------
\pagestyle{fancy}
\renewcommand{\headrulewidth}{0pt}

\fancyhead{}
\fancyhead[L]{%
\begin{minipage}[c]{1.6cm}
\includegraphics[width=1.5cm]{png/logo_ptsi.png}%
\end{minipage}
\rule{2cm}{.5pt}
}

\fancyhead[C]{\rule{11cm}{.5pt}}

\fancyhead[R]{%
\begin{minipage}[c]{3cm}
\begin{flushright}
\footnotesize{\textit{\textsf{Sciences Industrielles\\ de l'Ingénieur}}}%
\end{flushright}
\end{minipage}
}

\renewcommand{\footrulewidth}{0.2pt}

\fancyfoot[C]{\footnotesize{\bfseries \thepage}}
\fancyfoot[L]{\footnotesize{2012 -- 2013} \\ X. \textsc{Pessoles}}
\ifthenelse{\boolean{prof}}{%
\fancyfoot[R]{\footnotesize{TD -- CI 4 -- Conception des mécanismes -- P}}
}{%
\fancyfoot[R]{\footnotesize{TD -- CI 4 -- Conception des mécanismes}}
}


%\begin{center}
%\textit{Centre d'intérêt}
%\end{center}


\begin{center}
 \huge\textsc{CI 4 -- Conception des mécanismes}
\end{center}


\begin{center}
 \large\textsc{Conception d'un réducteur}
\end{center}
\vspace{.25cm}

\begin{flushright}
\textit{D'après ressources de Jean-Pierre Pupier.}
\end{flushright}
%
%\begin{minipage}[c]{.45\linewidth}
%\begin{center}
%\includegraphics[height=3cm]{png/moteur}
%
%\textit{Moteur de modélisme}
%\end{center}
%\end{minipage}\hfill
%\begin{minipage}[c]{.45\linewidth}
%\begin{center}
%\includegraphics[height=4cm]{png/moteur_3D}
%
%\textit{Représentation d'un moteur de modélisme}
%\end{center}
%\end{minipage}

\begin{contexte}
\begin{itemize}
\item Objectif pédagogique : concevoir un mécanisme à partir d'un cahier des charges
\item Objectif technique : 
\begin{itemize}
\item Concevoir un réducteur
\end{itemize}
\end{itemize}
\end{contexte}



\begin{minipage}[c]{.43\linewidth}

\section*{Présentation}
\subsection*{Analyse fonctionnelle technique}

Afin de réduire la vitesse de sortie d'un moteur électrique, on désire concevoir un réducteur à engrenage. L'entreprise prévoit de vendre un volume de 5 réducteurs par an. 

\end{minipage}\hfill
\begin{minipage}[c]{.43\linewidth}
\begin{center}
\includegraphics[width=.85\textwidth]{png/img1}
\end{center}
\end{minipage}

\begin{center}
\includegraphics[width=.9\textwidth]{png/img2}
\end{center}


Une étude préliminaire a conduit à se tourner vers l'architecture suivante. 

\vspace{.25cm}

\begin{minipage}[c]{.4\linewidth}
\begin{center}
\includegraphics[width=.8\textwidth]{png/img3}
\end{center}
\end{minipage}\hfill
\begin{minipage}[c]{.55\linewidth}
\begin{center}
\includegraphics[width=.95\textwidth]{png/img4}
\end{center}
\end{minipage}



\begin{center}
\includegraphics[width=.9\textwidth]{png/img5}
\end{center}



\begin{minipage}[c]{.47\linewidth}

Le carter sera réalisé en fonderie au sable. Il sera conçu en 2 parties conformément au schéma ci-contre.
\end{minipage}\hfill
\begin{minipage}[c]{.47\linewidth}
\begin{center}
\includegraphics[width=.85\textwidth]{png/img6}
\end{center}

\end{minipage}

\subsection*{Travail à réaliser}
On demande de réaliser : 
\begin{itemize}
\item le réducteur complet en coupe passant par le plan contenant les 2 axes; 
\item la mise en place de tous les ajustements notamment au niveau des roulements, des joints, de l'assemblage des carters;
\item d'indiquer les jeux fonctionnels;
\item d'indiquer le niveau d'huile au repos ainsi que les éléments permettant l'alimentation et la vidange. 
\end{itemize}

Conseils de conception : 
\begin{itemize}
\item l'arbre d'entrée est monté en porte a faux;
\item le pignon d'entrée est démontable car acheté sur catalogue (forme libre);
\item la couronne d'entrée est également achetée;
\item le support de couronne est en liaison à appui plan prépondérant avec l'arbre;
\item l'entraînement en rotation se fait par obstacle.
\end{itemize}

\end{document}