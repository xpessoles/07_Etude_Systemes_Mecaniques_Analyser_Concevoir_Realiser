\documentclass[11pt,oneside]{article}
\input{coursHeadings}
\usepackage[final]{pdfpages} 
\usepackage[%
    pdftitle={TD Conception},
    pdfauthor={Xavier Pessoles},
    colorlinks=true,
    linkcolor=blue,
    citecolor=magenta]{hyperref}



% \makeatletter \let\ps@plain\ps@empty \makeatother
%% DEBUT DU DOCUMENT
%% =================
\sloppy
\hyphenpenalty 10000

\newcommand{\Pointilles}[1][3]{%
\multido{}{#1}{\makebox[\linewidth]{\dotfill}\\[\parskip]
}}


\begin{document}


\newboolean{prof}
\setboolean{prof}{false}
%------------- En tetes et Pieds de Pages ------------
\pagestyle{fancy}
\renewcommand{\headrulewidth}{0pt}

\fancyhead{}
\fancyhead[L]{%
\begin{minipage}[c]{1.6cm}
\includegraphics[width=1.5cm]{png/logo_ptsi.png}%
\end{minipage}
\rule{2cm}{.5pt}
}

\fancyhead[C]{\rule{11cm}{.5pt}}

\fancyhead[R]{%
\begin{minipage}[c]{3cm}
\begin{flushright}
\footnotesize{\textit{\textsf{Sciences Industrielles\\ de l'Ingénieur}}}%
\end{flushright}
\end{minipage}
}

\renewcommand{\footrulewidth}{0.2pt}

\fancyfoot[C]{\footnotesize{\bfseries \thepage}}
\fancyfoot[L]{\footnotesize{2012 -- 2013} \\ X. \textsc{Pessoles}}
\ifthenelse{\boolean{prof}}{%
\fancyfoot[R]{\footnotesize{TD -- CI 4 -- Conception des mécanismes -- P}}
}{%
\fancyfoot[R]{\footnotesize{TD -- CI 4 -- Conception des mécanismes}}
}


%\begin{center}
%\textit{Centre d'intérêt}
%\end{center}


\begin{center}
 \huge\textsc{CI 4 -- Conception des mécanismes}
\end{center}


\begin{center}
 \large\textsc{Frein hydraulique d'un treuil}
\end{center}
\vspace{.5cm}

\begin{flushright}
\textit{D'après ressources de Jean-Pierre Pupier.}
\end{flushright}
%
%\begin{minipage}[c]{.45\linewidth}
%\begin{center}
%\includegraphics[height=3cm]{png/moteur}
%
%\textit{Moteur de modélisme}
%\end{center}
%\end{minipage}\hfill
%\begin{minipage}[c]{.45\linewidth}
%\begin{center}
%\includegraphics[height=4cm]{png/moteur_3D}
%
%\textit{Représentation d'un moteur de modélisme}
%\end{center}
%\end{minipage}

\begin{contexte}
\begin{itemize}
\item Objectif pédagogique : concevoir un mécanisme à partir d'un cahier des charges
\item Objectif technique : 
\begin{itemize}
\item Proposer une solution technologique permettant d'assurer le freinage de l'arbre d'un treuil hydraulique
\end{itemize}
\end{itemize}
\end{contexte}

\section*{Présentation}
\begin{minipage}[c]{.4\linewidth}

\begin{center}
\includegraphics[width=.5\textwidth]{png/treuil}

Treuil hydraulique Huchez TH -- 700 kg à 3,2 tonnes
\footnote{\url{http://www.huchez.fr/}}
\end{center}

\end{minipage}\hfill
\begin{minipage}[c]{.55\linewidth}
Les treuils hydrauliques sont utilisés par les entreprises ayant la nécessité de soulever de très grosses charges. Ils peuvent par exemple équiper les grues ou les camions de manutention. Un extrait du cahier des charges est donné ci-dessous.
\end{minipage}

\begin{center}
\includegraphics[width=.95\textwidth]{png/cdc}
\end{center}

Afin de satisfaire à la sécurité des personnes et des biens, on s'intéresse à la réalisation de la fonction FC1. Le système de freinage doit permettre de freiner le treuil lorsque l'opérateur le désire. Par ailleurs, en cas de défaillance du système hydraulique, on souhaite que, par défaut, le treuil soit freiné. Il s'agit ici de concevoir cette fonction.


L'appareil étudié comprend quatre modules : 
\begin{itemize}
\item moteur hydraulique à pistons axiaux;
\item réducteur à engrenages;
\item tambour (5);
\item frein à disque fixé en bout d'arbre et destiné à bloquer le tambour.
\end{itemize}


\section*{Travail demandé}



\paragraph*{}
\textit{A partir du schéma technologique fourni on demande de concevoir le frein qui permet le blocage du treuil en l'absence de pression.}

\begin{center}
\includegraphics[width=.65\textwidth]{png/schema_techno}
\end{center}

\section*{Caractéristiques techniques}

\begin{minipage}[c]{.47\linewidth}

\textbf{Moteur hydraulique :}
\begin{itemize}
\item Pression d'alimentation : $p_1 = 145 \; bars$
\item Fréquence de rotation : $N_1 = 1500 \; tr/min$
\end{itemize}


\textbf{Réducteur -- Pignons droits : }
\begin{itemize}
\item $Z_1 = 11$;
\item $Z_4" = 15$;
\item $Z_4' = 13$;
\item $Z_9 = 39$.
\end{itemize}

\textbf{Tambour : }
\begin{itemize}
\item câble (sur trois couches) : $\phi = 10,5\;mm$;
\item diamètre mini du tambour : $D_2 = 182\; mm$;
\item effort de traction maximal : $F_5 = 1\,500 \; daN$.
\end{itemize}


\end{minipage}\hfill
\begin{minipage}[c]{.47\linewidth}

\textbf{Frein à disque : }
\begin{itemize}
\item diamètre extérieur des surfaces frottantes : $D_e = 134\; mm$;
\item diamètre intérieur des surfaces frottantes : $D_i = 78\; mm$;
\item facteur de frottement : $f = 0,3$;
\item diamètre extérieur du piston 10 : $d_e = 100 \; mm$;
\item diamètre intérieur du piston 10 : $d_i = 74 \;mm$;
\item course du piston 10 avec garnitures neuves 11 : $1 \;mm$;
\item usure possible de chaque garniture 11 : $2\; mm$;
\item dimensions des rondelles Belleville formant le ressort 12 : $63 x 31 x 2$;
\item rigidité (constante) d'une rondelle : $4064 \; N / mm$;
\item course d'une rondelle : $1,8 \; mm$.
\end{itemize}

\end{minipage}



\begin{center}
\rotatebox{90}{\includegraphics[width=\textheight]{png/Plan}}
\end{center}



\section*{Critères d'évaluation}

Au vu du schéma technologique, les fonctions suivantes sont à réaliser :
\begin{itemize}
\item liaison pivot de l'arbre 1 avec le bâti. Cette liaison permet d'assurer le guidage de l'arbre à son extrémité gauche;
\item liaison pivot glissant du piston 10 avec le bâti. Le piston permettant de desserrer le frein hydraulique;
\item liaison glissière des disques 11 par rapport à l'arbre 1;
\item liaison glissière de la plaque 14 par rapport à 3;
\item le système de freinage comprenant l'arrivée du fluide sous pression ($P_{10}$), la compression des ressorts 12, l'étanchéité au niveau du piston.
\end{itemize}

Comme pour tout système mécanique, il faut prendre garde : 
\begin{itemize}
\item à ce que le système s'assemble;
\item à ce que les pièces puissent se fabriquer (par moulage, forgeage, usinage);
\item à préciser les ajustements;
\item à ce que le tracé soit soigné.
\end{itemize}

\end{document}