\documentclass[11pt,oneside]{article}
\input{coursHeadings}
\usepackage[final]{pdfpages} 
\usepackage[%
    pdftitle={TD Conception},
    pdfauthor={Xavier Pessoles},
    colorlinks=true,
    linkcolor=blue,
    citecolor=magenta]{hyperref}



% \makeatletter \let\ps@plain\ps@empty \makeatother
%% DEBUT DU DOCUMENT
%% =================
\sloppy
\hyphenpenalty 10000

\newcommand{\Pointilles}[1][3]{%
\multido{}{#1}{\makebox[\linewidth]{\dotfill}\\[\parskip]
}}


\begin{document}


\newboolean{prof}
\setboolean{prof}{false}
%------------- En tetes et Pieds de Pages ------------
\pagestyle{fancy}
\renewcommand{\headrulewidth}{0pt}

\fancyhead{}
\fancyhead[L]{%
\begin{minipage}[c]{1.6cm}
\includegraphics[width=2cm]{png/logo_ptsi.png}%
\end{minipage}
\rule{2cm}{.5pt}
}

\fancyhead[C]{\rule{11cm}{.5pt}}

\fancyhead[R]{%
\begin{minipage}[c]{3cm}
\begin{flushright}
\footnotesize{\textit{\textsf{Sciences Industrielles\\ pour l'Ingénieur}}}%
\end{flushright}
\end{minipage}
}

\renewcommand{\footrulewidth}{0.2pt}

\fancyfoot[C]{\footnotesize{\bfseries \thepage}}
\fancyfoot[L]{\footnotesize{2012 -- 2013} \\ X. \textsc{Pessoles}}
\ifthenelse{\boolean{prof}}{%
\fancyfoot[R]{\footnotesize{TD -- CI 4 -- Conception des mécanismes -- P}}
}{%
\fancyfoot[R]{\footnotesize{TD -- CI 4 -- Conception des mécanismes}}
}


%\begin{center}
%\textit{Centre d'intérêt}
%\end{center}


\begin{center}
 \huge\textsc{CI 4 -- Conception des mécanismes}
\end{center}

\begin{center}
 \LARGE\textsc{Conception des liaisons encastrement démontable}
\end{center}

\begin{center}
 \large\textsc{Moteur de modélisme}
\end{center}
\vspace{.5cm}


\begin{minipage}[c]{.45\linewidth}
\begin{center}
\includegraphics[height=3cm]{png/moteur}

\textit{Moteur de modélisme}
\end{center}
\end{minipage}\hfill
\begin{minipage}[c]{.45\linewidth}
\begin{center}
\includegraphics[height=4cm]{png/moteur_3D}

\textit{Représentation d'un moteur de modélisme}
\end{center}
\end{minipage}

\begin{contexte}
\begin{itemize}
\item Objectif pédagogique : concevoir une liaison encastrement démontable
\item Objectif technique : 
\begin{itemize}
\item Proposer des solutions technologiques qui permettent de raccorder l'hélice et l'arbre moteur
\item Proposer des solutions technologiques qui permettent d'assurer l'étanchéité entre le moteur et le milieu extérieur
\end{itemize}
\end{itemize}
\end{contexte}

\subsection*{Fonction globale}
Le mécanisme étudié est un moteur de modélisme. Le fonctionnement est celui d'un moteur classique, mais les choix technologiques (matériaux et composants) diffèrent de moteurs pouvant fournir davantage de puissance.

\begin{center}
\includegraphics[width=12cm]{png/af}
\end{center}


\subsubsection*{Conception de la liaison entre l'arbre et l'hélice}
\begin{minipage}[c]{.6\linewidth}
\paragraph{}
\textit{Réaliser la liaison entre l'arbre et l'hélice en utilisant un assemblage claveté ainsi qu'une vis de serrage.}

\paragraph{}
\textit{Réaliser la liaison entre l'arbre et l'hélice en utilisant un assemblage claveté ainsi qu'un écrou à encoche.}

\paragraph{}
\textit{Réaliser la liaison entre l'arbre et l'hélice en utilisant un assemblage claveté à cône prépondérant ainsi qu'un écrou.}

\paragraph{}
\textit{Préciser la nature des ajustements dans chacun des cas.}

\end{minipage}\hfill
\begin{minipage}[c]{.3\linewidth}
\begin{center}
\includegraphics[width=.8\textwidth]{png/helice}
\end{center}
\end{minipage}



\subsubsection*{Conception de la fermeture du carter}

On souhaite réaliser la fermeture du carter de moteur grâce à un flasque et un chapeau. Coté droit, le flasque devra permettre d'assurer l'étanchéité entre le vilebrequin et le carter par l'utilisation d'un joint à lèvre. Coté gauche, le chapeau devra inclure un dispositif permettant de réaliser une étanchéité statique.

\paragraph{}
\textit{Concevoir le flasque permettant la fermeture du carter coté droit. Préciser la nature des ajustements. (Serré ou glissant)}
\paragraph{}
\textit{Concevoir le flasque permettant la fermeture du carter coté gauche.}

%\includepdf{MoteurModelisme}


\end{document}