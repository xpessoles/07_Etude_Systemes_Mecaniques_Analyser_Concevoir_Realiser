\documentclass[11pt,oneside]{article}
\input{coursHeadings}
%\usepackage[raccourcis]{FAST}
\usepackage[%
    pdftitle={Produits Procédés Matériaux -- },
    pdfauthor={Xavier Pessoles},
    colorlinks=true,
    linkcolor=blue,
    citecolor=magenta]{hyperref}

\usepackage{pifont}


% \makeatletter \let\ps@plain\ps@empty \makeatother
%% DEBUT DU DOCUMENT
%% =================
\sloppy
\hyphenpenalty 10000

\newcommand{\Pointilles}[1][3]{%
\multido{}{#1}{\makebox[\linewidth]{\dotfill}\\[\parskip]
}}


\colorlet{shadecolor}{orange!15}

\newtheorem{theorem}{Theorem}


\begin{document}


\newboolean{prof}
\setboolean{prof}{true}
%------------- En tetes et Pieds de Pages ------------
\pagestyle{fancy}
\renewcommand{\headrulewidth}{0pt}

\fancyhead{}
\fancyhead[L]{%
\noindent\noindent\begin{minipage}[c]{2.6cm}
%Lycée Rouvière PTSI
\includegraphics[width=2cm]{png/logo_ptsi.png}%
\end{minipage}
}

\fancyhead[C]{\rule{12cm}{.5pt}}

\fancyhead[R]{%
\noindent\begin{minipage}[c]{3cm}
\begin{flushright}
\footnotesize{\textit{\textsf{Sciences Industrielles\\ pour l'Ingénieur}}}%
\end{flushright}
\end{minipage}
}

\renewcommand{\footrulewidth}{0.2pt}

\fancyfoot[C]{\footnotesize{\bfseries \thepage}}
\fancyfoot[L]{\footnotesize{2012 -- 2013} \\ X. \textsc{Pessoles}}
\ifthenelse{\boolean{prof}}{%
\fancyfoot[R]{\footnotesize{TD -- CI 6 : PPM -- P}}
}{%
\fancyfoot[R]{\footnotesize{Cours -- CI 6 : PPM}}
}



\begin{center}
 \huge\textsc{CI 6 -- PPM -- Produits Procédés Matériaux}

 \large\textsc{Élaboration des pièces mécaniques. Introduction de la chaîne numérique.}
\end{center}

%\begin{center}
% \LARGE\textsc{Chapitre 5 -- Procédé de moulage -- Conception des pièces moulées}
%\end{center}

%\begin{flushright}
%\textit{D'après documents de Jean-Pierre Pupier}
%\end{flushright}

\vspace{.5cm}




\begin{contexte}
\begin{itemize}
\item Objectif pédagogique : lire les spécifications géométriques sur un dessin d'ensemble
\item Objectif technique : 
\begin{itemize}
\item %Quelles seront les formes finales du carter intermédiaire selon que celui-ci sera réalisé en moulage ou en mécano soudage ?
\end{itemize}
\end{itemize}
\end{contexte}

\subsection*{Kart Speed O Max}
La pièce étudiée est un moyeu de volant de Kart, qui équipe des karts grand
public à motorisation électrique, conçus par la société Speed O Max.

Le moyeu est la pièce qui positionne le volant par rapport au Kart. Il est
nécessaire que ec volant soit parfaitement disposé par rapport au kart,
notamment en ligne droite, pour permettre une bonne
perception de la trajectoire par le pilote.

\begin{center}
 \begin{tabular}{cc}
  \includegraphics[height=4cm]{png/kart1}&
\includegraphics[height=4cm]{png/kart2}
 \end{tabular}
\end{center}

Le dessin de définition ci-après propose une solution pour ce moyeu.

\paragraph{}
\textit{Que signifie l'indication \textbf{ISO 2768 mK} ? Vous pourrez pour cela utiliser le GDI (Chapitre 16 -- Inscription des tolérances).}

\begin{minipage}[c]{.45\linewidth}
\begin{center}
\includegraphics[width=.9\textwidth]{png/kart3}
\end{center}
\end{minipage} \hfill
\begin{minipage}[c]{.45\linewidth}
\paragraph{}
\textit{Expliciter la désignation \textbf{AlCuMg4}.} 

\paragraph{}
\textit{Donner les différentes opérations qui ont mené à la fabrication du moyeu.}

\paragraph{}
\textit{Expliciter les spécifications de perpenducularité.}

\paragraph{}
\textit{Expliciter les spécifications de localisation. Vous commencerez par celles qui ont le moins d'éléments de référence.}


\end{minipage}


\end{document}