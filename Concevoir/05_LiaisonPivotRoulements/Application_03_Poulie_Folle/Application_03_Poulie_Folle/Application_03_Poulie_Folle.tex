\documentclass[10pt]{article}
\input{style/coursHeadings}
\input{style/programHeadings}
\input{style/macros_SII}
\input{style/macros_Titres}
\input{style/macros_Frames}

%Si le boolen xp est vrai : compilation pour xabi
%Sinon compilation Damien
\newboolean{xp}
\setboolean{xp}{true}

\newboolean{prof}
\setboolean{prof}{false}

\newif\ifprof
%\proftrue
\proffalse

\newboolean{td}
\setboolean{td}{true}

\usepackage[%
    pdftitle={CI7 - CPT},
    pdfauthor={Xavier Pessoles},
    colorlinks=true,
    linkcolor=blue,
    citecolor=magenta]{hyperref}

\def\discipline{Sciences Industrielles de l'Ingénieur}

\def\xxtitre{\ifthenelse{\boolean{xp}}{CI 7 -- CPT : Étude des systèmes mécaniques -- Analyser, Concevoir, Réaliser
}{}}

\def\xxsoustitre{\ifthenelse{\boolean{xp}}{
Chapitre 5 -- Liaison Pivot - Éléments roulants}{
}}


\def\xxauteur{\ifthenelse{\boolean{xp}}{
\noindent Xavier \textsc{Pessoles}}{
}}


\def\xxpied{\ifthenelse{\boolean{xp}}{
CI 7 : Systèmes mécaniques -- Conception\\
Ch 5 : Liaison Pivot -- Éléments Roulants -- Application -- \ifthenelse{\boolean{prof}}{P}{E}%
}{
}}

\def\xxcathegorie{\ifthenelse{\boolean{xp}}{
}{
Informatique - Cours}}

%---------------------------------------------------------------------------


\begin{document}

\ifthenelse{\boolean{xp}}{\input{style/enteteXP}}{\input{style/enteteDI}}

\begin{flushright}
\textit{D'après ressources de Maryline Carrez.}
\end{flushright}


%\renewcommand{\baselinestretch}{1.2}
%\setlength{\parskip}{2ex plus 0.5ex minus 0.2ex}


%
%\begin{comp}
%\noindent \textbf{Résoudre :} à partir des modèles retenus :
%\begin{itemize}
%\item choisir une méthode de résolution analytique, graphique, numérique;
%\item mettre en \oe{}uvre une méthode de résolution.
%\end{itemize}
%
%\noindent \textit{Rés -- C1.1 :} Loi entrée sortie géométrique et cinématique -- Fermeture géométrique.
%
%\end{comp}



\section*{Poulie folle}

Il s’agit de concevoir un système permettant de réaliser une
« poulie folle », à savoir un mécanisme de renvoi de tension.
L’ensemble est composé d’une poulie 1 (fonte EN-GJS 350-5),
d’un axe 2 (acier E335), d’une équerre moulée 3 (acier GS 400),
d’une tige verticale 4 (acier E335) et d’un support 5 (fonte EN-
GJS 350-5).
La charge extérieure est constante et est due principalement à la
tension de la courroie. La direction de la courroie reste fixe.
L’étanchéité dynamique sera réalisée par des joints à une lèvre
et l’étanchéité statique par joint toriques.
On donne le schéma technologique. La tige 4 est en liaison pivot
avec le support vertical 5 et la poulie 1 est en liaison pivot avec
l’équerre 3.


\begin{minipage}[c]{.47\linewidth}
\begin{center}
\includegraphics[width=.4\textwidth]{images/Fig1}
\end{center}
\end{minipage} \hfill
\begin{minipage}[c]{.47\linewidth}
\begin{center}
\includegraphics[width=.4\textwidth]{images/Fig2}
\end{center}
\end{minipage}

\subparagraph{}\textit{Concevoir la liaison pivot entre la poulie 1 et l’équerre 3 en utilisant des roulements à billes à contact radial.}
%\subparagraph{}\textit{Concevoir la liaison encastrement démontable entre l’équerre 3 et l’axe 1.}
\subparagraph{}\textit{Concevoir la liaison entre le support 5 et l’équerre 3 via la tige 4.}
\subparagraph{}\textit{Préciser les ajustements ainsi que les jeux fonctionnels.}


\begin{center}
\includegraphics[height=\textheight]{images/Fig3}
\end{center}
\end{document}

