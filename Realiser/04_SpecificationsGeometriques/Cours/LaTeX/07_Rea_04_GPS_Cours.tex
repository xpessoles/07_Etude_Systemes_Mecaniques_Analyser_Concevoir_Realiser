\documentclass[10pt]{article}
\input{style/coursHeadings}
\input{style/programHeadings}
\input{style/macros_SII}
\input{style/macros_Titres}
\input{style/macros_Frames}

%Si le boolen xp est vrai : compilation pour xabi
%Sinon compilation Damien

\newif\ifprof
%\proftrue
\proffalse

\newif\ifxp
\xptrue
%\xpfalse

\newif\iftd
\tdtrue
%\tdfalse


\usepackage[%
    pdftitle={},
    pdfauthor={Xavier Pessoles},
    colorlinks=true,
    linkcolor=blue,
    citecolor=magenta]{hyperref}


\def\discipline{Sciences Industrielles de l'Ingénieur}
\def\xxtitre{%
\ifxp
7 : Étude des systèmes mécaniques : Analyser -- Concevoir -- Réaliser
\else
\fi
}

\def\xxsoustitre{%
\ifxp
Réaliser -- Chapitre 4 -- Spécification Géométrique des Produits
\else
\fi}

\def\xxauteur{%
\ifxp
Xavier \textsc{Pessoles}
\else
\fi}

\def\xxpied{%
\ifxp
7 : Étude des systèmes mécaniques : Analyser -- Concevoir -- Réaliser\\
Réaliser -- Chapitre 4 -- Spécification Géométrique des Produits -- Cours
\else
\fi}



%---------------------------------------------------------------------------


\begin{document}

\input{style/enteteXP}


\setlength{\parskip}{0ex plus 0.2ex minus 0ex}
 \renewcommand{\contentsname}{}
 \renewcommand{\baselinestretch}{1}

\tableofcontents

 \renewcommand{\baselinestretch}{1.2}
\setlength{\parskip}{2ex plus 0.5ex minus 0.2ex}



\section{Spécification des états de surface}
\subsection{Origine des défauts d'ordre 1 à 4}
\subsubsection{Défauts d'ordre 1 : défauts géométriques}

\subsubsection{Défauts d'ordre 2 : défauts d'ondulation}

\subsubsection{Défauts d'ordre 3 : défauts de rugosité}

\subsubsection{Défauts d'ordre 4 : défauts d'arrachement}

\section{Définitions}
\subsection{Notion de filtrage}

\begin{minipage}[c]{.2\linewidth}
\begin{center}
\end{center}
\end{minipage}\hfill
\begin{minipage}[c]{.75\linewidth}
\begin{defi}
\textbf{Direction du profil} 

\end{defi}
\end{minipage}


\begin{minipage}[c]{.2\linewidth}
\begin{center}
\end{center}
\end{minipage}\hfill
\begin{minipage}[c]{.75\linewidth}
\begin{defi}
\textbf{Longueur de base -- longueur d'évaluation} 

\end{defi}
\end{minipage}


\begin{minipage}[c]{.2\linewidth}
\begin{center}
\end{center}
\end{minipage}\hfill
\begin{minipage}[c]{.75\linewidth}
\begin{defi}
\textbf{Ligne de référence moyenne} 

La ligne de référence est déterminée en ne prenant compte que les défauts de rugosité. Elle est définie telle que l'aire comprise entre le profil et la ligne est égale de part et d'autre de ce profil.
\end{defi}
\end{minipage}


\subsection{Paramètres de rugosité}


\subsubsection{Rugosité totale} 
\begin{minipage}[c]{.2\linewidth}
\begin{center}
\end{center}
\end{minipage}\hfill
\begin{minipage}[c]{.75\linewidth}
\begin{defi}
\textbf{Rugosité totale -- $R_T$ :}

Hauteur entre le plus bas des creux et la plus haut des pics.

\end{defi}
\end{minipage}

\subsubsection{Rugosité maximale} 
\begin{minipage}[c]{.2\linewidth}
\begin{center}
\end{center}
\end{minipage}\hfill
\begin{minipage}[c]{.75\linewidth}
\begin{defi}
\textbf{Rugosité maximale -- $R_{max j}$} 

Hauteur entre un pic et un creux successifs.

\textbf{Rugosité maximale -- $R_{max}$} 

Pour un profil donné, plus grand $R_{max j}$
\end{defi}
\end{minipage}

\subsubsection{Rugosité arithmétique} 

\begin{thebibliography}{2}
\bibitem{jpp}{Jean-Pierre Pupier, Fabrication Mécanique, PT, Lycée Rouvière, Toulon.}
\end{thebibliography}
\end{document}


