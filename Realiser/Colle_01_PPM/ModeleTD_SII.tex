\documentclass[10pt,fleqn]{article} % Default font size and left-justified equations
\usepackage[%
    pdftitle={Cycle 5 : PPM},
    pdfauthor={Xavier Pessoles}]{hyperref}
    
\input{style/new_style}
\input{style/macros_SII}

\usepackage{multicol}
\fichetrue
%\fichefalse

%\proftrue
\proffalse

\tdtrue
%\tdfalse

\courstrue
\coursfalse

\def\discipline{Sciences \\Industrielles de \\ l'Ingénieur}
\def\xxtete{Sciences Industrielles de l'Ingénieur}

\def\classe{PTSI}
\def\xxnumpartie{Cycle 5}
\def\xxpartie{Produits -- Matériaux -- Procédés \\
Analyser, Réaliser, Concevoir}

\def\xxnumchapitre{Chapitres 1, 2, 3}
\def\xxchapitre{}

\def\xxtitreexo{Exercices d'application du cours}
\def\xxsourceexo{\hspace{.2cm} }


\def\xxposongletx{2}
\def\xxposonglettext{1.45}
\def\xxposonglety{20}
\def\xxonglet{Cy. 5 -- Ch. 123}

\def\xxactivite{Colle 1}
\def\xxauteur{\textsl{Xavier Pessoles}}

\def\xxcompetences{%
\textsl{%
%\textbf{Savoirs et compétences :}\\
%%\noindent \textbf{Résoudre :} à partir des modèles retenus :
%\begin{itemize}[label=\ding{112},font=\color{ocre}] 
%%\item %\textit{Rés -- C1.1 :} Loi entrée sortie géométrique et cinématique -- Fermeture géométrique.
%\item Écrire le vecteur position, vitesse d’un point d’un solide.
%\item Écrire le vecteur accélération d’un point d’un solide.
%\end{itemize}
%
%\noindent \textit{Mod2 -- C4.1 :} Représentation par schéma bloc.
}}

\def\xxfigures{
%\includegraphics[width=.8\textwidth]{images/prot_01}
}%figues de la page de garde

\def\xxpied{%
Partie 5 -- Produits -- Matériaux -- Procédés  \\
Ch. 1, 2 et 3 -- \xxactivite%
}


\setcounter{secnumdepth}{5}
%---------------------------------------------------------------------------


\begin{document}
%\chapterimage{images/Fond_Cin}
\input{style/new_pagegarde}
\vspace{8cm}
\pagestyle{fancy}
\thispagestyle{plain}


\def\columnseprulecolor{\color{ocre}}
\setlength{\columnseprule}{0.4pt} 

\section*{Matériaux et Procédés}
\begin{multicols}{2}
\subparagraph{}
\textit{Donner le nom de 4 familles de matériaux}

\subparagraph{}
\textit{Donner la définition d’un acier et d’une fonte.}

\subparagraph{}
\textit{Donner le nom d’un critère de dureté. }


\subparagraph{}
\textit{Qu’est-ce que la résilience ? Par quel essai est-elle caractérisée ?}

On donne ci-dessous les courbes d'un essai de traction. 
\subparagraph{}
\textit{Quelles doivent être l’abscisse et l’ordonnée d’un essai normalisé ?}

\subparagraph{}
\textit{Pour le StE 600 identifier la zone élastique et la zone plastique. }


\subparagraph{}
\textit{Pour le StE 600 donner la résistance élastique, la résistance maximale à la rupture et le module de Young.}

\end{multicols}
\begin{center}
\includegraphics[width=.8\textwidth]{images/courbe}
\end{center}
%\newpage
%\setcounter{exo}{0}
%\section*{Matériaux et Procédés}
%\begin{multicols}{2}
%\end{multicols}

\end{document}


